\chapter{Discussion, Conclusions and Future Work}

\section{Discussion}

As mentioned in the introduction, one of the objectives in developing this system was that it could work with emerging technology in the market capable of capturing the movements of a user, so that it would be an economically viable system, which notably restricts the range of possibilities to be considered. The configuration chosen for the system uses six sensors attached to the user's body provided by \texttt{VR}. This limits the system's ability to perceive the most complex movements in the case of using two base stations, which are responsible for collecting information from the six optical sensors. There may be cases in which torso rotations are not correctly detected, which can generate confusion between different members of the avatar that have axial symmetry (after the rotation, the system confuses the exact position of each of the components). It is expected in the future to be able to extend this work with the use of a greater number of sensors and more modern virtual reality glasses.

On the other hand, some of the works studied (e.g. \cite{Keerthy:Thesis:2012,Kyan:2015:ABD:2753829.2735951}) use more sophisticated techniques to analyze the movements of the users than those used in this work. While such techniques can be useful at advanced learning levels, when it comes to beginners in capoeira, there are no major differences from more modest techniques. This is because the movements of an apprentice are simpler and therefore the recording of the trainer's movements is also simpler. Therefore, a simpler method of recording movements has two clear advantages: first, it is less complex and faster to calculate; and, second, the result of the analysis is more easily translatable into a system like augmented reality, providing the apprentice with useful clarifications about how to improve the performance of the techniques.

\section{Conclusions}

Throughout the present work, we have described the process by which a virtual capoeira trainer has been developed from the previous recording of an expert's movements made with the \textit{VR} device. On the other hand, the user, making use of the \textit{Smartphone} will make an analysis of the movements captured by the teacher, which will serve to identify possible errors in their execution.

Once the transitions have been recorded, an application is offered that is capable of visualizing the movements that were executed incorrectly. In this process we must emphasize the ability of text devices to capture the movements correctly, a relevant fact if we also take into account that we are before a leading technology on the market, and at a moderate cost.

The benefits of using a virtual trainer should be highlighted, ranging from the comfort of working from home to having a dedicated, personal and immediate text feedback. Similarly, it is a training system through game, which enhances learning and motivation, as seen in the scene of the student, being able to interact with the real world. It allows to establish the desired difficulty to record the movements, which makes it possible to adapt to the needs and capacities of the different students. It is an intuitive environment for both the teacher and the student. 

Although an extensive evaluation has not yet been carried out with both novice and expert capoeira users, preliminary tests have been conducted to check the correct functioning of the applications and the reliability of the feedback from users who have used them to correct their shortcomings. These tests were conducted with a small sample of users at different levels, and the preliminary results show that, at the level at which the tests were conducted and with not excessively complex whole-body movements, the operation of the applications is fast and the results of the analyses and possible observations of errors are quite close to reality.

\section{Future work}

This project has potential in the field of physical activity training, since comparing the movement of the user with that of an expert offers a visual \textit{feedback}. In this way, the user will be able to correct his technique in case he makes any kind of mistake. 

A functionality that can be implemented in the future would be the recording and comparison of movements in real time between the teacher and the student. When the student puts on another virtual reality lens and the sensors on the body, they could even fight in the same virtual zone, but each in a completely different location

Another future extension tries to add more functionality to the expert who records the movements. With this, he would have the possibility to handle and modify the recorded movements in a simple way, similar to the design made in this project, but without the need to be in the pc, but inside the virtual reality.

On the other hand, it is necessary to include more sophisticated methods for the analysis of the movements, similar to those used in \cite{Keerthy:Thesis:2012,Kyan:2015:ABD:2753829.2735951} to allow a more refined analysis of the movements that provides more useful results for advanced capoeira practitioners.

This movement analysis procedure can also be applied in the medical field to perform different types of rehabilitation. In this case, it would allow the patient to analyze the movements that he cannot do, such as a leg injury, so that by observing the movements that he should perform, he would stimulate his brain to perform them correctly.

Another factor for improving movement capture would be to add more base stations, in order to achieve greater precision in the execution of the movements collected by the sensors, thus making movement capture worthy of a professional system.

Finally, the recording and analysis of the movements of more than one user simultaneously is also contemplated. In this way, situations similar to the work in pairs that takes place in a 'capoeira' ring, where hand-to-hand combat techniques are put into practice, can be reproduced.
