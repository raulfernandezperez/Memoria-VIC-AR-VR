\chapter{Distribución del trabajo}
En una primera instancia fue necesario realizar una investigación sobre el campo de trabajo. El cual comprende el uso de la informática y nuevas tecnólogas aplicadas a la captura de movimiento, que, al ser un tema totalmente novedoso para el desarrollador, implicó un amplio estudio del estado del arte actual.  

El objetivo general del proyecto era el desarrollo de un entrenador personal virtual basado en el arte marcial afro-brasileño capoeira. Para ello, la intención consista en mostrar, mediante un escenario 3D, el entrenamiento realizado por un usuario para el aprendizaje de capoeira, siendo necesario exponer los movimientos del entrenador virtual con el fin de ser imitando. Inicialmente, se desconocía como se iba a llevar a cabo este objetivo global y abstracto, ya que requería la comprensión profunda de dispositivos de realidad virtual y realidad aumentada, para descubrir el abanico de posibilidades disponibles. Uno de los aspectos más importantes del proyecto ha sido todo lo relacionado con la captura de movimientos. 

Tras un breve periodo inicial de aprendizaje en algunas partes de \textit{Unity}, el desarrollador contaban con los conocimientos necesarios para abordar el escenario 3D. Una vez entendido el funcionamiento de la simulación de la cinemática inversa, se investigó la manera de guardar la información de la captura de movimiento, empleando así, sistemas para realizar grabaciones a partir de decenas de datos en \textit{Unity}. 

Los datos se archivan guardando la identificación y los datos de los seis puntos de seguimiento existentes en el cuerpo humanoide y cada una de las coordenadas(X,Y,Z)  asociadas a las articulaciones. Gracias a la investigación anterior, se diseñó un sistema capaz de reproducir los movimiento pregrabados, realizados por el desarrollador, a la vez que se simulaba el movimiento en tiempo real de un usuario. Esto consiste en generar los datos del usuario registrados en las diferentes fuentes de los sensores. La primera de ellas, trata los datos obtenidos a partir de la sensores de \textit{VR} y la segunda, genera el posterior almacenamiento mediante archivos de datos de \textit{Unity}.

Llegados a este punto, se estudió el uso de utilizar diferentes tecnologías capaces de desempeñar una de las funcionalidades del entrenador virtual, el análisis de los movimientos. El planteamiento exitoso, consistía en crear un sistema capaz de analizar el movimiento, y este a su vez, desempeñar un comportamiento de máquina de estados, sirviendo el primer estado para calibrar la posición inicial del movimiento y los siguientes estados se encargan de analizar el movimiento a entrenar. 

La siguiente etapa consistía en conectar los huesos de los avatares con la información de los sensores ofrecidos por \textit{VR}, de esta manera, los avatares son animados por los movimientos del profesor. 

Paralelamente, se investigó como darles vida a los personajes del proyecto, con el objetivo de desarrollar diferentes avatares con una estética humana. Se crearon varios avatares, pero el inicial fue muy sencillo, ya que de momento interesaba ver el resultado que proporcionaba esta aplicación junto con la de realidad aumentada. De los diferentes avatares creados, uno de ellos fue el que se utilizó en un principio, pero al observar que este avatar no plasmaba el realismo esperado, se decidió buscar otro diseño que pudiera realizar esta tarea. 

Con toda esta información, se diseñaron los avatares del proyecto, como son el alumno, profesor. Además, diseño la ropa típica de capoeira para los personajes, como son el pantalón largo y la sudadera de color blanco. Teniendo ya todo el desarrollo avanzado, se iniciaron los correspondientes procesos para desplegar la aplicación de realidad aumentada con \textit{ARCORE} y la de realidad virtual con \textit{SteamVR}.

El desarrollo de toda esta memoria ha sido realizado con \textit{Latex}, ya que se trata de un sistema de composición de textos, orientado a la creación de documentos escritos que presenten una alta calidad tipográfica.
