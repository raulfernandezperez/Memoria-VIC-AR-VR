%---------------------------------------------------------------------
%
%                          introduction.tex
%
%---------------------------------------------------------------------


\addtocounter{chapter}{-1} 
\chapter{Introduction}

\label{cap1:sec:introduction}

Over the years, the human movement has been the subject of numerous studies and research mostly in the area of medicine, sports science and biomedical engineering. Given the great amount of technologies available in the market, the need to take these studies to other areas in constant growth arises, giving rise to the new era of motion capture systems. At present, these systems have become an essential tool in industries as important as the film and video game industries. These systems called \textit{Mocap} greatly facilitate the work of animators and developers in bringing the movements of characters to life. Given the great technological advance that these systems have experienced, they have managed to generate incredibly realistic and precise movements, which considerably reduce the production costs and development times necessary to bring large \textit{films} or video game titles to life. 

This way of giving life to 3D models can be particularly useful not only in the industries mentioned above, but also in the creation of systems capable of diagnosing medical treatments, analysis of athletes in sports medicine, motor recovery in the abilities of disabled people, study of movements made by machines, etc.

So the Reactive Virtual Trainers (RVT) come into play, it is a technology that has been used for some years for dance, martial arts and physical exercises in general. Taking into account these functionalities and the great capacities that technology provides us, it was decided to join the training of the Brazilian martial art, known as capoeira, with a system of movement capture in \textit{VR}, giving rise to the creation and development of a virtual personal trainer capable of teaching this sport. 

Capoeira uses one of the fundamental characteristics used by many martial arts; it is the repeated performance of the same movement in order to gradually improve its execution, either in its speed, strength or trajectory. In the case of beginners in this type of martial arts, it is quite complex to appreciate the existence of a possible improvement in the execution of the movements without the constant supervision of a trainer, which is very complicated in most cases. Therefore, the need arises to solve this existing lack with the development of this project, through the creation of a personal trainer of capoeira.

After observing the studies related to VTRs, they determined that users pay more attention to systems capable of showing movements of objects in 3D, ambient music and the different colors that it provides us with to illustrate all this on a screen, unlike an everyday personal trainer. For this reason, in all the studies carried out on VTX, a great evolution has been observed, since there is a competition to try to perform the movements with a more and more refined technique. Another important advantage in the realization of this project, is the capacity for immersion that provide us with technologies such as virtual reality and augmented reality, capable of immersing us in environments completely adaptable to each of the needs given, without having to move from our home, residence or hospital, for the realization of a rehabilitation, among others. Therefore, it is possible to use them for all types of audiences without having to move to the place where these activities would normally be performed.

Therefore, the goal of this project is to capture the movements made by a capoeira teacher. Obtaining the data in real time from six sensors placed on his body, capable of capturing the transition of his movements through the technology \textit{Mocap} in \textit{VR}. For the analysis and continuous learning that the student will do about capoeira, the technology of \textit{AR} will be used. With this system, it is intended to use in an innovative way the \textit{RVT} systems to visualize the movements through the camera of a \textit{smartphone}, observing the real world at the same time that the transitions of the movements done by the teacher are analyzed in a more effective, close and visual way. In order to achieve a more realistic and dynamic environment, different avatars were developed with a completely human appearance. In addition to all this, two different scenarios were designed, one for the field of \textit{VR}, staging an environment for the creation and reproduction of the movements made by the teacher. A second scenario, created and designed for \textit{AR} capable of observing the sequence of movements made by the teacher in a close, analytical and visual of each of the transitions made. With all this, it was possible to create two applications with which the user felt comfortable being in a more real environment.


\section{Work Objectives}

In the development of textit{RVT} systems using \textit{Mocap} they focus essentially on how to make a realistic representation of how a user can train and perfect different techniques in different physical activities, in this case in the Brazilian martial art (capoeira). To this end, these systems are based on the repeated performance of a series of movements, comparing the technique performed by themselves and the movements recorded by an expert. Following these guidelines, some objectives were determined to be fulfilled in this End-of-Grade Work (TFG):

\begin{itemize}
    \item Identify the limitations of VR devices to achieve correct and effective motion capture on hoods.
    \item Design and develop two applications with the help of a \textit{Game Engine} capable of representing a virtual trainer using augmented reality and virtual reality techniques.
    \item Determine an analysis of the capoeira movements captured with the VR-text sensors and classify and represent by means of AR-text which were the best recorded movements for the training.
\end{itemize}


\section{Project/Work Plan}

The present project has been elaborated in three different phases: specification, development and documentation of the project.

In the first phase, the objectives to be dealt with and the scope of the TFG were established, and it was necessary to organize dates to arrange meetings with the tutors and thus determine the follow-up of the development of the project.

Later, in the second phase, the specifications set out in the previous phase are put into practice. This phase is in turn divided into different steps: the first of these was the choice of the environment to be used, so that it would be suitable for the technology used in 'Textit' and 'AR'. The second step consisted of developing two clearly differentiable applications, both of which encompass a 3D environment, capable of using logic to capture movements in real time, represented by different avatars. Finally, after finishing the development of the applications, the recordings and adjustments of the recorded movements were made with experts in capoeira, analyzing and adjusting the movements that would later be incorporated into the personal trainer.

Finally, in the documentation phase of the project, a research process is carried out on all the existing information on motion capture and the content necessary for the elaboration of the End of Degree Project. This phase was executed in parallel with the development of the applications.


\section{Memory Structure}


The following project is structured in six chapters defined as follows:

\begin{itemize}
    \item Chapter one describes the introduction to the project, the main objectives and the work plan.
    \item Chapter two describes the history of motion capture and the different methods of motion capture on the market.
    \item Chapter three covers the technologies used, both hardware and software, and justifies the decision on the environments used.
    \item Chapter four describes the entire development of the project, explaining in different sections the previous study for its creation and the necessary implementations to give shape and life to the virtual capoeira coach. In addition, the transition of the different scenarios of the application of \textit{VR} and \textit{AR} is explained.
    \item Chapter five presents the discussions, the conclusion drawn from the project and the proposals for the future.
    \item Chapter six describes the distribution and organization of the work done.
\end{itemize}
