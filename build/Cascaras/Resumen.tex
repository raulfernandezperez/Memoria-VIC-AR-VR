%---------------------------------------------------------------------
%
%                      resumen.tex
%
%---------------------------------------------------------------------


\chapter*{Resumen}
\addcontentsline{toc}{chapter}{Resumen}

Este proyecto tiene como meta el diseño, desarrollo y despliegue de un entrenador personal virtual mediante la captura de movimiento utilizando la tecnología de realidad virtual y realidad aumentada.
Con el objetivo de entender y abordar de una manera correcta este proyecto, se han revisador diferentes tecnologías existentes en el mercado para el desarrollo de un sistema capaz de captura los movimientos en tiempo real. Se han expuesto las ventajas que han llevado a utilizar dispositivos de \textit{VR}, mencionando diferentes estudios previos con resultados satisfactorios. Además, se han detallado las novedades que aportan las diferentes soluciones existentes en la danza, los entrenadores personales o las artes marciales, en el ámbito de los \textit{Reactive Virtual Trainers (RVT)}.
Este proyecto presenta un sistema para el entrenamiento personal de un arte marcial brasileño (capoeira), el cual se define como una mezcla entre los deportes de danza y lucha. El entorno desarrollado está pensado para que el alumno pueda practicar y perfeccionar diferentes movimientos sobre capoeira sin la necesidad de que un profesor revise sus movimientos.
Para la realización del entrenamiento, es necesario imitar una serie de movimientos grabados previamente por expertos en el arte marcial. Para que el alumno pueda avanzar en la realización de los movimientos, existen diferentes niveles de entrenamiento donde el alumno va aprendiendo de forma progresiva.
El sistema es capaz de adaptarse fácilmente para, por ejemplo, la medicina deportiva o la rehabilitación de alguna parte del cuerpo dañada.
Tras el estudio realizado, se pudo observar que el funcionamiento de las aplicaciones es inmersivo, intuitivo y rápido, ofreciendo un sistema amplio para 
determinar de una forma visual cuales son los movimientos ejecutados de forma errónea. Por todo ello, se puede afirmar que este tipo de sistemas son el presente y futuro para el aprendizaje en diferentes disciplinas.




Palabras clave:Mocap, captura de movimiento, Reactive Virtual Trainer, RVT, Unity, entrenador personal, Vr y AR.
